\chapter{More about Types}
\label{ch:more-types}

So far we have learned about a few basic types (for example \isa{bool} and
\isa{nat}), type abbreviations (\isacommand{types}) and recursive datatypes
(\isacommand{datatype}). This chapter will introduce more
advanced material:
\begin{itemize}
\item Pairs ({\S}\ref{sec:products}) and records ({\S}\ref{sec:records}),
and how to reason about them.
\item Type classes: how to specify and reason about axiomatic collections of
  types ({\S}\ref{sec:axclass}). This section leads on to a discussion of
  Isabelle's numeric types ({\S}\ref{sec:numbers}).  
\item Introducing your own types: how to define types that
  cannot be constructed with any of the basic methods
  ({\S}\ref{sec:adv-typedef}).
\end{itemize}

The material in this section goes beyond the needs of most novices.
Serious users should at least skim the sections as far as type classes.
That material is fairly advanced; read the beginning to understand what it
is about, but consult the rest only when necessary.

\index{pairs and tuples|(}
\input{Pairs}    %%%Section "Pairs and Tuples"
\index{pairs and tuples|)}

\input{Records}  %%%Section "Records"


\section{Type Classes} %%%Section
\label{sec:axclass}
\index{axiomatic type classes|(}
\index{*axclass|(}

The programming language Haskell has popularized the notion of type
classes: a type class is a set of types with a
common interface: all types in that class must provide the functions
in the interface.  Isabelle offers a similar type class concept: in
addition, properties (\emph{class axioms}) can be specified which any
instance of this type class must obey.  Thus we can talk about a type
$\tau$ being in a class $C$, which is written $\tau :: C$.  This is the case
if $\tau$ satisfies the axioms of $C$. Furthermore, type classes can be
organized in a hierarchy.  Thus there is the notion of a class $D$
being a subclass\index{subclasses} of a class $C$, written $D
< C$. This is the case if all axioms of $C$ are also provable in $D$.

In this section we introduce the most important concepts behind type
classes by means of a running example from algebra.  This should give
you an intuition how to use type classes and to understand
specifications involving type classes.  Type classes are covered more
deeply in a separate tutorial \cite{isabelle-classes}.

\subsection{Overloading}
\label{sec:overloading}
\index{overloading|(}

\input{Overloading}

\index{overloading|)}

\input{Axioms}

\index{type classes|)}
\index{*class|)}

\section{Numbers}
\label{sec:numbers}

\index{numbers|(}%
Until now, our numerical examples have used the type of \textbf{natural
numbers},
\isa{nat}.  This is a recursive datatype generated by the constructors
zero  and successor, so it works well with inductive proofs and primitive
recursive function definitions.  HOL also provides the type
\isa{int} of \textbf{integers}, which lack induction but support true
subtraction.  With subtraction, arithmetic reasoning is easier, which makes
the integers preferable to the natural numbers for
complicated arithmetic expressions, even if they are non-negative.  There are also the types
\isa{rat}, \isa{real} and \isa{complex}: the rational, real and complex numbers.  Isabelle has no 
subtyping,  so the numeric
types are distinct and there are functions to convert between them.
Most numeric operations are overloaded: the same symbol can be
used at all numeric types. Table~\ref{tab:overloading} in the appendix
shows the most important operations, together with the priorities of the
infix symbols. Algebraic properties are organized using type classes
around algebraic concepts such as rings and fields;
a property such as the commutativity of addition is a single theorem 
(\isa{add_commute}) that applies to all numeric types.

\index{linear arithmetic}%
Many theorems involving numeric types can be proved automatically by
Isabelle's arithmetic decision procedure, the method
\methdx{arith}.  Linear arithmetic comprises addition, subtraction
and multiplication by constant factors; subterms involving other operators
are regarded as variables.  The procedure can be slow, especially if the
subgoal to be proved involves subtraction over type \isa{nat}, which 
causes case splits.  On types \isa{nat} and \isa{int}, \methdx{arith}
can deal with quantifiers---this is known as Presburger arithmetic---whereas on type \isa{real} it cannot.

The simplifier reduces arithmetic expressions in other
ways, such as dividing through by common factors.  For problems that lie
outside the scope of automation, HOL provides hundreds of
theorems about multiplication, division, etc., that can be brought to
bear.  You can locate them using Proof General's Find
button.  A few lemmas are given below to show what
is available.

\subsection{Numeric Literals}
\label{sec:numerals}

\index{numeric literals|(}%
The constants \cdx{0} and \cdx{1} are overloaded.  They denote zero and one,
respectively, for all numeric types.  Other values are expressed by numeric
literals, which consist of one or more decimal digits optionally preceeded by a minus sign (\isa{-}).  Examples are \isa{2}, \isa{-3} and
\isa{441223334678}.  Literals are available for the types of natural
numbers, integers, rationals, reals, etc.; they denote integer values of
arbitrary size.

Literals look like constants, but they abbreviate 
terms representing the number in a two's complement binary notation. 
Isabelle performs arithmetic on literals by rewriting rather 
than using the hardware arithmetic. In most cases arithmetic 
is fast enough, even for numbers in the millions. The arithmetic operations 
provided for literals include addition, subtraction, multiplication, 
integer division and remainder.  Fractions of literals (expressed using
division) are reduced to lowest terms.

\begin{warn}\index{overloading!and arithmetic}
The arithmetic operators are 
overloaded, so you must be careful to ensure that each numeric 
expression refers to a specific type, if necessary by inserting 
type constraints.  Here is an example of what can go wrong:
\par
\begin{isabelle}
\isacommand{lemma}\ "2\ *\ m\ =\ m\ +\ m"
\end{isabelle}
%
Carefully observe how Isabelle displays the subgoal:
\begin{isabelle}
\ 1.\ (2::'a)\ *\ m\ =\ m\ +\ m
\end{isabelle}
The type \isa{'a} given for the literal \isa{2} warns us that no numeric
type has been specified.  The problem is underspecified.  Given a type
constraint such as \isa{nat}, \isa{int} or \isa{real}, it becomes trivial.
\end{warn}

\begin{warn}
\index{function@\isacommand {function} (command)!and numeric literals}  
Numeric literals are not constructors and therefore
must not be used in patterns.  For example, this declaration is
rejected:
\begin{isabelle}
\isacommand{function}\ h\ \isakeyword{where}\isanewline
"h\ 3\ =\ 2"\isanewline
\isacharbar "h\ i\ \ =\ i"
\end{isabelle}

You should use a conditional expression instead:
\begin{isabelle}
"h\ i\ =\ (if\ i\ =\ 3\ then\ 2\ else\ i)"
\end{isabelle}
\index{numeric literals|)}
\end{warn}


\subsection{The Type of Natural Numbers, {\tt\slshape nat}}

\index{natural numbers|(}\index{*nat (type)|(}%
This type requires no introduction: we have been using it from the
beginning.  Hundreds of theorems about the natural numbers are
proved in the theories \isa{Nat} and \isa{Divides}.  
Basic properties of addition and multiplication are available through the
axiomatic type class for semirings (\S\ref{sec:numeric-classes}).

\subsubsection{Literals}
\index{numeric literals!for type \protect\isa{nat}}%
The notational options for the natural  numbers are confusing.  Recall that an
overloaded constant can be defined independently for each type; the definition
of \cdx{1} for type \isa{nat} is
\begin{isabelle}
1\ \isasymequiv\ Suc\ 0
\rulename{One_nat_def}
\end{isabelle}
This is installed as a simplification rule, so the simplifier will replace
every occurrence of \isa{1::nat} by \isa{Suc\ 0}.  Literals are obviously
better than nested \isa{Suc}s at expressing large values.  But many theorems,
including the rewrite rules for primitive recursive functions, can only be
applied to terms of the form \isa{Suc\ $n$}.

The following default  simplification rules replace
small literals by zero and successor: 
\begin{isabelle}
2\ +\ n\ =\ Suc\ (Suc\ n)
\rulename{add_2_eq_Suc}\isanewline
n\ +\ 2\ =\ Suc\ (Suc\ n)
\rulename{add_2_eq_Suc'}
\end{isabelle}
It is less easy to transform \isa{100} into \isa{Suc\ 99} (for example), and
the simplifier will normally reverse this transformation.  Novices should
express natural numbers using \isa{0} and \isa{Suc} only.

\subsubsection{Division}
\index{division!for type \protect\isa{nat}}%
The infix operators \isa{div} and \isa{mod} are overloaded.
Isabelle/HOL provides the basic facts about quotient and remainder
on the natural numbers:
\begin{isabelle}
m\ mod\ n\ =\ (if\ m\ <\ n\ then\ m\ else\ (m\ -\ n)\ mod\ n)
\rulename{mod_if}\isanewline
m\ div\ n\ *\ n\ +\ m\ mod\ n\ =\ m%
\rulenamedx{mod_div_equality}
\end{isabelle}

Many less obvious facts about quotient and remainder are also provided. 
Here is a selection:
\begin{isabelle}
a\ *\ b\ div\ c\ =\ a\ *\ (b\ div\ c)\ +\ a\ *\ (b\ mod\ c)\ div\ c%
\rulename{div_mult1_eq}\isanewline
a\ *\ b\ mod\ c\ =\ a\ *\ (b\ mod\ c)\ mod\ c%
\rulename{mod_mult_right_eq}\isanewline
a\ div\ (b*c)\ =\ a\ div\ b\ div\ c%
\rulename{div_mult2_eq}\isanewline
a\ mod\ (b*c)\ =\ b * (a\ div\ b\ mod\ c)\ +\ a\ mod\ b%
\rulename{mod_mult2_eq}\isanewline
0\ <\ c\ \isasymLongrightarrow \ (c\ *\ a)\ div\ (c\ *\ b)\ =\ a\ div\ b%
\rulename{div_mult_mult1}\isanewline
(m\ mod\ n)\ *\ k\ =\ (m\ *\ k)\ mod\ (n\ *\ k)
\rulenamedx{mod_mult_distrib}\isanewline
m\ \isasymle \ n\ \isasymLongrightarrow \ m\ div\ k\ \isasymle \ n\ div\ k%
\rulename{div_le_mono}
\end{isabelle}

Surprisingly few of these results depend upon the
divisors' being nonzero.
\index{division!by zero}%
That is because division by
zero yields zero:
\begin{isabelle}
a\ div\ 0\ =\ 0
\rulename{DIVISION_BY_ZERO_DIV}\isanewline
a\ mod\ 0\ =\ a%
\rulename{DIVISION_BY_ZERO_MOD}
\end{isabelle}
In \isa{div_mult_mult1} above, one of
the two divisors (namely~\isa{c}) must still be nonzero.

The \textbf{divides} relation\index{divides relation}
has the standard definition, which
is overloaded over all numeric types: 
\begin{isabelle}
m\ dvd\ n\ \isasymequiv\ {\isasymexists}k.\ n\ =\ m\ *\ k
\rulenamedx{dvd_def}
\end{isabelle}
%
Section~\ref{sec:proving-euclid} discusses proofs involving this
relation.  Here are some of the facts proved about it:
\begin{isabelle}
\isasymlbrakk m\ dvd\ n;\ n\ dvd\ m\isasymrbrakk \ \isasymLongrightarrow \ m\ =\ n%
\rulenamedx{dvd_antisym}\isanewline
\isasymlbrakk k\ dvd\ m;\ k\ dvd\ n\isasymrbrakk \ \isasymLongrightarrow \ k\ dvd\ (m\ +\ n)
\rulenamedx{dvd_add}
\end{isabelle}

\subsubsection{Subtraction}

There are no negative natural numbers, so \isa{m\ -\ n} equals zero unless 
\isa{m} exceeds~\isa{n}. The following is one of the few facts
about \isa{m\ -\ n} that is not subject to
the condition \isa{n\ \isasymle \  m}. 
\begin{isabelle}
(m\ -\ n)\ *\ k\ =\ m\ *\ k\ -\ n\ *\ k%
\rulenamedx{diff_mult_distrib}
\end{isabelle}
Natural number subtraction has few
nice properties; often you should remove it by simplifying with this split
rule.
\begin{isabelle}
P(a-b)\ =\ ((a<b\ \isasymlongrightarrow \ P\
0)\ \isasymand \ (\isasymforall d.\ a\ =\ b+d\ \isasymlongrightarrow \ P\
d))
\rulename{nat_diff_split}
\end{isabelle}
For example, splitting helps to prove the following fact.
\begin{isabelle}
\isacommand{lemma}\ "(n\ -\ 2)\ *\ (n\ +\ 2)\ =\ n\ *\ n\ -\ (4::nat)"\isanewline
\isacommand{apply}\ (simp\ split:\ nat_diff_split,\ clarify)\isanewline
\ 1.\ \isasymAnd d.\ \isasymlbrakk n\ <\ 2;\ n\ *\ n\ =\ 4\ +\ d\isasymrbrakk \ \isasymLongrightarrow \ d\ =\ 0
\end{isabelle}
The result lies outside the scope of linear arithmetic, but
 it is easily found
if we explicitly split \isa{n<2} as \isa{n=0} or \isa{n=1}:
\begin{isabelle}
\isacommand{apply}\ (subgoal_tac\ "n=0\ |\ n=1",\ force,\ arith)\isanewline
\isacommand{done}
\end{isabelle}%%%%%%
\index{natural numbers|)}\index{*nat (type)|)}


\subsection{The Type of Integers, {\tt\slshape int}}

\index{integers|(}\index{*int (type)|(}%
Reasoning methods for the integers resemble those for the natural numbers, 
but induction and
the constant \isa{Suc} are not available.  HOL provides many lemmas for
proving inequalities involving integer multiplication and division, similar
to those shown above for type~\isa{nat}. The laws of addition, subtraction
and multiplication are available through the axiomatic type class for rings
(\S\ref{sec:numeric-classes}).

The \rmindex{absolute value} function \cdx{abs} is overloaded, and is 
defined for all types that involve negative numbers, including the integers.
The \isa{arith} method can prove facts about \isa{abs} automatically, 
though as it does so by case analysis, the cost can be exponential.
\begin{isabelle}
\isacommand{lemma}\ "abs\ (x+y)\ \isasymle \ abs\ x\ +\ abs\ (y\ ::\ int)"\isanewline
\isacommand{by}\ arith
\end{isabelle}

For division and remainder,\index{division!by negative numbers}
the treatment of negative divisors follows
mathematical practice: the sign of the remainder follows that
of the divisor:
\begin{isabelle}
0\ <\ b\ \isasymLongrightarrow \ 0\ \isasymle \ a\ mod\ b%
\rulename{pos_mod_sign}\isanewline
0\ <\ b\ \isasymLongrightarrow \ a\ mod\ b\ <\ b%
\rulename{pos_mod_bound}\isanewline
b\ <\ 0\ \isasymLongrightarrow \ a\ mod\ b\ \isasymle \ 0
\rulename{neg_mod_sign}\isanewline
b\ <\ 0\ \isasymLongrightarrow \ b\ <\ a\ mod\ b%
\rulename{neg_mod_bound}
\end{isabelle}
ML treats negative divisors in the same way, but most computer hardware
treats signed operands using the same rules as for multiplication.
Many facts about quotients and remainders are provided:
\begin{isabelle}
(a\ +\ b)\ div\ c\ =\isanewline
a\ div\ c\ +\ b\ div\ c\ +\ (a\ mod\ c\ +\ b\ mod\ c)\ div\ c%
\rulename{zdiv_zadd1_eq}
\par\smallskip
(a\ +\ b)\ mod\ c\ =\ (a\ mod\ c\ +\ b\ mod\ c)\ mod\ c%
\rulename{mod_add_eq}
\end{isabelle}

\begin{isabelle}
(a\ *\ b)\ div\ c\ =\ a\ *\ (b\ div\ c)\ +\ a\ *\ (b\ mod\ c)\ div\ c%
\rulename{zdiv_zmult1_eq}\isanewline
(a\ *\ b)\ mod\ c\ =\ a\ *\ (b\ mod\ c)\ mod\ c%
\rulename{zmod_zmult1_eq}
\end{isabelle}

\begin{isabelle}
0\ <\ c\ \isasymLongrightarrow \ a\ div\ (b*c)\ =\ a\ div\ b\ div\ c%
\rulename{zdiv_zmult2_eq}\isanewline
0\ <\ c\ \isasymLongrightarrow \ a\ mod\ (b*c)\ =\ b*(a\ div\ b\ mod\
c)\ +\ a\ mod\ b%
\rulename{zmod_zmult2_eq}
\end{isabelle}
The last two differ from their natural number analogues by requiring
\isa{c} to be positive.  Since division by zero yields zero, we could allow
\isa{c} to be zero.  However, \isa{c} cannot be negative: a counterexample
is
$\isa{a} = 7$, $\isa{b} = 2$ and $\isa{c} = -3$, when the left-hand side of
\isa{zdiv_zmult2_eq} is $-2$ while the right-hand side is~$-1$.
The prefix~\isa{z} in many theorem names recalls the use of $\mathbb{Z}$ to
denote the set of integers.%
\index{integers|)}\index{*int (type)|)}

Induction is less important for integers than it is for the natural numbers, but it can be valuable if the range of integers has a lower or upper bound.  There are four rules for integer induction, corresponding to the possible relations of the bound ($\geq$, $>$, $\leq$ and $<$):
\begin{isabelle}
\isasymlbrakk k\ \isasymle \ i;\ P\ k;\ \isasymAnd i.\ \isasymlbrakk k\ \isasymle \ i;\ P\ i\isasymrbrakk \ \isasymLongrightarrow \ P(i+1)\isasymrbrakk \ \isasymLongrightarrow \ P\ i%
\rulename{int_ge_induct}\isanewline
\isasymlbrakk k\ <\ i;\ P(k+1);\ \isasymAnd i.\ \isasymlbrakk k\ <\ i;\ P\ i\isasymrbrakk \ \isasymLongrightarrow \ P(i+1)\isasymrbrakk \ \isasymLongrightarrow \ P\ i%
\rulename{int_gr_induct}\isanewline
\isasymlbrakk i\ \isasymle \ k;\ P\ k;\ \isasymAnd i.\ \isasymlbrakk i\ \isasymle \ k;\ P\ i\isasymrbrakk \ \isasymLongrightarrow \ P(i-1)\isasymrbrakk \ \isasymLongrightarrow \ P\ i%
\rulename{int_le_induct}\isanewline
\isasymlbrakk i\ <\ k;\ P(k-1);\ \isasymAnd i.\ \isasymlbrakk i\ <\ k;\ P\ i\isasymrbrakk \ \isasymLongrightarrow \ P(i-1)\isasymrbrakk \ \isasymLongrightarrow \ P\ i%
\rulename{int_less_induct}
\end{isabelle}


\subsection{The Types of Rational, Real and Complex Numbers}
\label{sec:real}

\index{rational numbers|(}\index{*rat (type)|(}%
\index{real numbers|(}\index{*real (type)|(}%
\index{complex numbers|(}\index{*complex (type)|(}%
These types provide true division, the overloaded operator \isa{/}, 
which differs from the operator \isa{div} of the 
natural numbers and integers. The rationals and reals are 
\textbf{dense}: between every two distinct numbers lies another.
This property follows from the division laws, since if $x\not=y$ then $(x+y)/2$ lies between them:
\begin{isabelle}
a\ <\ b\ \isasymLongrightarrow \ \isasymexists r.\ a\ <\ r\ \isasymand \ r\ <\ b%
\rulename{dense}
\end{isabelle}

The real numbers are, moreover, \textbf{complete}: every set of reals that
is bounded above has a least upper bound.  Completeness distinguishes the
reals from the rationals, for which the set $\{x\mid x^2<2\}$ has no least
upper bound.  (It could only be $\surd2$, which is irrational.) The
formalization of completeness, which is complicated, 
can be found in theory \texttt{RComplete}.

Numeric literals\index{numeric literals!for type \protect\isa{real}}
for type \isa{real} have the same syntax as those for type
\isa{int} and only express integral values.  Fractions expressed
using the division operator are automatically simplified to lowest terms:
\begin{isabelle}
\ 1.\ P\ ((3\ /\ 4)\ *\ (8\ /\ 15))\isanewline
\isacommand{apply} simp\isanewline
\ 1.\ P\ (2\ /\ 5)
\end{isabelle}
Exponentiation can express floating-point values such as
\isa{2 * 10\isacharcircum6}, which will be simplified to integers.

\begin{warn}
Types \isa{rat}, \isa{real} and \isa{complex} are provided by theory HOL-Complex, which is
Main extended with a definitional development of the rational, real and complex
numbers.  Base your theory upon theory \thydx{Complex_Main}, not the
usual \isa{Main}.%
\end{warn}

Available in the logic HOL-NSA is the
theory \isa{Hyperreal}, which define the type \tydx{hypreal} of 
\rmindex{non-standard reals}.  These
\textbf{hyperreals} include infinitesimals, which represent infinitely
small and infinitely large quantities; they facilitate proofs
about limits, differentiation and integration~\cite{fleuriot-jcm}.  The
development defines an infinitely large number, \isa{omega} and an
infinitely small positive number, \isa{epsilon}.  The 
relation $x\approx y$ means ``$x$ is infinitely close to~$y$.''
Theory \isa{Hyperreal} also defines transcendental functions such as sine,
cosine, exponential and logarithm --- even the versions for type
\isa{real}, because they are defined using nonstandard limits.%
\index{rational numbers|)}\index{*rat (type)|)}%
\index{real numbers|)}\index{*real (type)|)}%
\index{complex numbers|)}\index{*complex (type)|)}


\subsection{The Numeric Type Classes}\label{sec:numeric-classes}

Isabelle/HOL organises its numeric theories using axiomatic type classes.
Hundreds of basic properties are proved in the theory \isa{Ring_and_Field}.
These lemmas are available (as simprules if they were declared as such)
for all numeric types satisfying the necessary axioms. The theory defines
dozens of type classes, such as the following:
\begin{itemize}
\item 
\tcdx{semiring} and \tcdx{ordered_semiring}: a \emph{semiring}
provides the associative operators \isa{+} and~\isa{*}, with \isa{0} and~\isa{1}
as their respective identities. The operators enjoy the usual distributive law,
and addition (\isa{+}) is also commutative.
An \emph{ordered semiring} is also linearly
ordered, with addition and multiplication respecting the ordering. Type \isa{nat} is an ordered semiring.
\item 
\tcdx{ring} and \tcdx{ordered_ring}: a \emph{ring} extends a semiring
with unary minus (the additive inverse) and subtraction (both
denoted~\isa{-}). An \emph{ordered ring} includes the absolute value
function, \cdx{abs}. Type \isa{int} is an ordered ring.
\item 
\tcdx{field} and \tcdx{ordered_field}: a field extends a ring with the
multiplicative inverse (called simply \cdx{inverse} and division~(\isa{/})).
An ordered field is based on an ordered ring. Type \isa{complex} is a field, while type \isa{real} is an ordered field.
\item 
\tcdx{division_by_zero} includes all types where \isa{inverse 0 = 0}
and \isa{a / 0 = 0}. These include all of Isabelle's standard numeric types.
However, the basic properties of fields are derived without assuming
division by zero.
\end{itemize}

Hundreds of basic lemmas are proved, each of which
holds for all types in the corresponding type class. In most
cases, it is obvious whether a property is valid for a particular type. No
abstract properties involving subtraction hold for type \isa{nat};
instead, theorems such as
\isa{diff_mult_distrib} are proved specifically about subtraction on
type~\isa{nat}. All abstract properties involving division require a field.
Obviously, all properties involving orderings required an ordered
structure.

The class \tcdx{ring_no_zero_divisors} of rings without zero divisors satisfies a number of natural cancellation laws, the first of which characterizes this class:
\begin{isabelle}
(a\ *\ b\ =\ (0::'a))\ =\ (a\ =\ (0::'a)\ \isasymor \ b\ =\ (0::'a))
\rulename{mult_eq_0_iff}\isanewline
(a\ *\ c\ =\ b\ *\ c)\ =\ (c\ =\ (0::'a)\ \isasymor \ a\ =\ b)
\rulename{mult_cancel_right}
\end{isabelle}
\begin{pgnote}
Setting the flag \pgmenu{Isabelle} $>$ \pgmenu{Settings} $>$
\pgmenu{Show Sorts} will display the type classes of all type variables.
\end{pgnote}
\noindent
Here is how the theorem \isa{mult_cancel_left} appears with the flag set.
\begin{isabelle}
((c::'a::ring_no_zero_divisors)\ *\ (a::'a::ring_no_zero_divisors) =\isanewline
\ c\ *\ (b::'a::ring_no_zero_divisors))\ =\isanewline
(c\ =\ (0::'a::ring_no_zero_divisors)\ \isasymor\ a\ =\ b)
\end{isabelle}


\subsubsection{Simplifying with the AC-Laws}
Suppose that two expressions are equal, differing only in 
associativity and commutativity of addition.  Simplifying with the
following equations sorts the terms and groups them to the right, making
the two expressions identical.
\begin{isabelle}
a\ +\ b\ +\ c\ =\ a\ +\ (b\ +\ c)
\rulenamedx{add_assoc}\isanewline
a\ +\ b\ =\ b\ +\ a%
\rulenamedx{add_commute}\isanewline
a\ +\ (b\ +\ c)\ =\ b\ +\ (a\ +\ c)
\rulename{add_left_commute}
\end{isabelle}
The name \isa{add_ac}\index{*add_ac (theorems)} 
refers to the list of all three theorems; similarly
there is \isa{mult_ac}.\index{*mult_ac (theorems)} 
They are all proved for semirings and therefore hold for all numeric types.

Here is an example of the sorting effect.  Start
with this goal, which involves type \isa{nat}.
\begin{isabelle}
\ 1.\ Suc\ (i\ +\ j\ *\ l\ *\ k\ +\ m\ *\ n)\ =\
f\ (n\ *\ m\ +\ i\ +\ k\ *\ j\ *\ l)
\end{isabelle}
%
Simplify using  \isa{add_ac} and \isa{mult_ac}.
\begin{isabelle}
\isacommand{apply}\ (simp\ add:\ add_ac\ mult_ac)
\end{isabelle}
%
Here is the resulting subgoal.
\begin{isabelle}
\ 1.\ Suc\ (i\ +\ (m\ *\ n\ +\ j\ *\ (k\ *\ l)))\
=\ f\ (i\ +\ (m\ *\ n\ +\ j\ *\ (k\ *\ l)))%
\end{isabelle}


\subsubsection{Division Laws for Fields}

Here is a selection of rules about the division operator.  The following
are installed as default simplification rules in order to express
combinations of products and quotients as rational expressions:
\begin{isabelle}
a\ *\ (b\ /\ c)\ =\ a\ *\ b\ /\ c
\rulename{times_divide_eq_right}\isanewline
b\ /\ c\ *\ a\ =\ b\ *\ a\ /\ c
\rulename{times_divide_eq_left}\isanewline
a\ /\ (b\ /\ c)\ =\ a\ *\ c\ /\ b
\rulename{divide_divide_eq_right}\isanewline
a\ /\ b\ /\ c\ =\ a\ /\ (b\ *\ c)
\rulename{divide_divide_eq_left}
\end{isabelle}

Signs are extracted from quotients in the hope that complementary terms can
then be cancelled:
\begin{isabelle}
-\ (a\ /\ b)\ =\ -\ a\ /\ b
\rulename{minus_divide_left}\isanewline
-\ (a\ /\ b)\ =\ a\ /\ -\ b
\rulename{minus_divide_right}
\end{isabelle}

The following distributive law is available, but it is not installed as a
simplification rule.
\begin{isabelle}
(a\ +\ b)\ /\ c\ =\ a\ /\ c\ +\ b\ /\ c%
\rulename{add_divide_distrib}
\end{isabelle}


\subsubsection{Absolute Value}

The \rmindex{absolute value} function \cdx{abs} is available for all 
ordered rings, including types \isa{int}, \isa{rat} and \isa{real}.
It satisfies many properties,
such as the following:
\begin{isabelle}
\isasymbar x\ *\ y\isasymbar \ =\ \isasymbar x\isasymbar \ *\ \isasymbar y\isasymbar 
\rulename{abs_mult}\isanewline
(\isasymbar a\isasymbar \ \isasymle \ b)\ =\ (a\ \isasymle \ b\ \isasymand \ -\ a\ \isasymle \ b)
\rulename{abs_le_iff}\isanewline
\isasymbar a\ +\ b\isasymbar \ \isasymle \ \isasymbar a\isasymbar \ +\ \isasymbar b\isasymbar 
\rulename{abs_triangle_ineq}
\end{isabelle}

\begin{warn}
The absolute value bars shown above cannot be typed on a keyboard.  They
can be entered using the X-symbol package.  In \textsc{ascii}, type \isa{abs x} to
get \isa{\isasymbar x\isasymbar}.
\end{warn}


\subsubsection{Raising to a Power}

Another type class, \tcdx{ordered\_idom}, specifies rings that also have 
exponentation to a natural number power, defined using the obvious primitive
recursion. Theory \thydx{Power} proves various theorems, such as the 
following.
\begin{isabelle}
a\ \isacharcircum \ (m\ +\ n)\ =\ a\ \isacharcircum \ m\ *\ a\ \isacharcircum \ n%
\rulename{power_add}\isanewline
a\ \isacharcircum \ (m\ *\ n)\ =\ (a\ \isacharcircum \ m)\ \isacharcircum \ n%
\rulename{power_mult}\isanewline
\isasymbar a\ \isacharcircum \ n\isasymbar \ =\ \isasymbar a\isasymbar \ \isacharcircum \ n%
\rulename{power_abs}
\end{isabelle}%%%%%%%%%%%%%%%%%%%%%%%%%
\index{numbers|)}
             %%%Section "Numbers"

\input{Typedefs}    %%%Section "Introducing New Types"
