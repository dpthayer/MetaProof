\chapter{Inductively Defined Sets} \label{chap:inductive}
\index{inductive definitions|(}

This chapter is dedicated to the most important definition principle after
recursive functions and datatypes: inductively defined sets.

We start with a simple example: the set of even numbers.  A slightly more
complicated example, the reflexive transitive closure, is the subject of
{\S}\ref{sec:rtc}. In particular, some standard induction heuristics are
discussed. Advanced forms of inductive definitions are discussed in
{\S}\ref{sec:adv-ind-def}. To demonstrate the versatility of inductive
definitions, the chapter closes with a case study from the realm of
context-free grammars. The first two sections are required reading for anybody
interested in mathematical modelling.

\begin{warn}
Predicates can also be defined inductively.
See {\S}\ref{sec:ind-predicates}.
\end{warn}

\input{Even}
\input{Mutual}
\input{Star}

\section{Advanced Inductive Definitions}
\label{sec:adv-ind-def}
\input{Advanced}

\input{AB}

\index{inductive definitions|)}
