
\documentclass[a4paper,fleqn]{article}

\usepackage{latexsym,graphicx}
\usepackage[refpage]{nomencl}
\usepackage{iman,extra,isar}
\usepackage{isabelle,isabellesym}
\usepackage{style}
\usepackage{mathpartir}
\usepackage{amsthm}
\usepackage{pdfsetup}

\newcommand{\cmd}[1]{\isacommand{#1}}

\newcommand{\isasymINFIX}{\cmd{infix}}
\newcommand{\isasymLOCALE}{\cmd{locale}}
\newcommand{\isasymINCLUDES}{\cmd{includes}}
\newcommand{\isasymDATATYPE}{\cmd{datatype}}
\newcommand{\isasymDEFINES}{\cmd{defines}}
\newcommand{\isasymNOTES}{\cmd{notes}}
\newcommand{\isasymCLASS}{\cmd{class}}
\newcommand{\isasymINSTANCE}{\cmd{instance}}
\newcommand{\isasymLEMMA}{\cmd{lemma}}
\newcommand{\isasymPROOF}{\cmd{proof}}
\newcommand{\isasymQED}{\cmd{qed}}
\newcommand{\isasymFIX}{\cmd{fix}}
\newcommand{\isasymASSUME}{\cmd{assume}}
\newcommand{\isasymSHOW}{\cmd{show}}
\newcommand{\isasymNOTE}{\cmd{note}}
\newcommand{\isasymCODEGEN}{\cmd{code\_gen}}
\newcommand{\isasymPRINTCODETHMS}{\cmd{print\_codethms}}
\newcommand{\isasymFUN}{\cmd{fun}}
\newcommand{\isasymFUNCTION}{\cmd{function}}
\newcommand{\isasymPRIMREC}{\cmd{primrec}}
\newcommand{\isasymRECDEF}{\cmd{recdef}}

\newcommand{\qt}[1]{``#1''}
\newcommand{\qtt}[1]{"{}{#1}"{}}
\newcommand{\qn}[1]{\emph{#1}}
\newcommand{\strong}[1]{{\bfseries #1}}
\newcommand{\fixme}[1][!]{\strong{FIXME: #1}}

\newtheorem{exercise}{Exercise}{\bf}{\itshape}
%\newtheorem*{thmstar}{Theorem}{\bf}{\itshape}

\hyphenation{Isabelle}
\hyphenation{Isar}

\isadroptag{theory}
\title{Defining Recursive Functions in Isabelle/HOL}
\author{Alexander Krauss}

\isabellestyle{tt}
\renewcommand{\isastyletxt}{\isastyletext}% use same formatting for txt and text

\begin{document}

\date{\ \\}
\maketitle

\begin{abstract}
  This tutorial describes the use of the new \emph{function} package,
	which provides general recursive function definitions for Isabelle/HOL.
	We start with very simple examples and then gradually move on to more
	advanced topics such as manual termination proofs, nested recursion,
	partiality, tail recursion and congruence rules.
\end{abstract}

%\thispagestyle{empty}\clearpage

%\pagenumbering{roman}
%\clearfirst

\section{Sorting}\label{Sorting}

\input{Functions.tex}
%\section{Conclusion}

\fixme{}






\begingroup
%\tocentry{\bibname}
\bibliographystyle{plain} \small\raggedright\frenchspacing
\bibliography{manual}
\endgroup

\end{document}


%%% Local Variables: 
%%% mode: latex
%%% TeX-master: t
%%% End: 
